\documentclass[11pt,a4paper]{article}
\usepackage{float}
\usepackage{verbatim}
\usepackage{subfig}
\usepackage[T1]{fontenc}
\usepackage[utf8]{inputenc}
\usepackage{geometry}
\geometry{verbose,lmargin=2cm,rmargin=2cm, bmargin=3cm, tmargin=3cm}
\usepackage{enumitem}
\usepackage{wrapfig}
\usepackage{tikz}
\usetikzlibrary{decorations.markings}
\usepackage{calc}
\usepackage{wrapfig}
\usepackage{graphicx}
\usepackage{amssymb}
\usepackage{amsmath}
\usepackage{esint}
\usepackage{blindtext}
\usepackage{hyperref}
\usepackage{listings}

\hypersetup{
    colorlinks=true,
    linkcolor=blue,
    filecolor=magenta,
    urlcolor=cyan,
}
\usepackage{listings}
\lstset{ %
  backgroundcolor=\color{white},   % choose the background color; you must add \usepackage{color} or \usepackage{xcolor}; should come as last argument
  basicstyle=\footnotesize,        % the size of the fonts that are used for the code
  breakatwhitespace=false,         % sets if automatic breaks should only happen at whitespace
  breaklines=true,                 % sets automatic line breaking
  captionpos=t,                    % sets the caption-position to bottom
  commentstyle=\color{teal},    % comment style
  deletekeywords={...},            % if you want to delete keywords from the given language
  escapeinside={\%*}{*)},          % if you want to add LaTeX within your code
  extendedchars=true,              % lets you use non-ASCII characters; for 8-bits encodings only, does not work with UTF-8
  frame=single,                    % adds a frame around the code
  keepspaces=true,                 % keeps spaces in text, useful for keeping indentation of code (possibly needs columns=flexible)
  keywordstyle=\color{blue},       % keyword style
 % language=Python,                 % the language of the code
  morekeywords={*,...},           % if you want to add more keywords to the set
  numbers=left,                    % where to put the line-numbers; possible values are (none, left, right)
  numbersep=5pt,                   % how far the line-numbers are from the code
  numberstyle=\tiny\color{black}, % the style that is used for the line-numbers
  rulecolor=\color{black},         % if not set, the frame-color may be changed on line-breaks within not-black text (e.g. comments (green here))
  showspaces=false,                % show spaces everywhere adding particular underscores; it overrides 'showstringspaces'
  showstringspaces=false,          % underline spaces within strings only
  showtabs=false,                  % show tabs within strings adding particular underscores
  stepnumber=1,                    % the step between two line-numbers. If it's 1, each line will be numbered
  tabsize=2,                       % sets default tabsize to 2 spaces
  title=\lstname                   % show the filename of files included with \lstinputlisting; also try caption instead of title
}
\begin{document}

\title{Polarization\\
\normalsize{FYS2150 Lab Report}\\}

\author{Nicholas Karlsen}
% \email{nichoka@student.matnat.uio.no}

\date{\today}% It is always \today, today,
%  but any date may be explicitly specified

\maketitle
\begin{abstract}
  Studying the properties of linearly and circularly polarized light and testing how well the properties match up to theoretical predictions.
\end{abstract}

\twocolumn

%\tableofcontents

\section{\label{sect:intro}Introduction}
  

\section{\label{sect:theory}Theory}

  \subsection{Birefringence}
    Certain materials, such as Calcite crystal exhibit birefringent behavior, where it effectively has two separate refractive indices for light polarized at right angle to each other, meaning that when unpolarized light is passed through the crystal, two displaced images will be visible. Birefringence can occur for different reasons, but in the case of crystalline Calcite, it is due to the uniaxial geometry of its lattice (See fig. \ref{fig:calcite}). Where light which is polarized is perpendicular to this optical axis will follow a certain, set refractive index, regardless of the crystals orientation. 

    \begin{figure}[H]
      \center
      \includegraphics[width=8cm]{scripts/figs/calcite.jpg}
      \caption{Crystaline structure of calcite (Source:\url{http://lecturedemo.ph.unimelb.edu.au/Optics/Crystal-optics/Oh-1-Calcite-Crystal-Model/Calcite-Diagram)}}
      \label{fig:calcite}
    \end{figure}

\section{\label{section:experimental}Experimental Procedure} 
  
  In order to minimize the effects that ambient light may have on the following experiments, the windows in the room were covered up and all lights not related to the experiment were turned off whilst data was being recorded, or observations were made.

  \subsection{\label{subsect:polar_lamp}Checking the polarization of the spectral lamp}
    In order to determine whether or not the light emitted from a specific Sodium spectral lamp, the intensity of light was measured using a luxmeter\cite{data:luxmeter} after having gone through a polarization filter of variable angle acting as an analyzer, depicted in Fig \ref{fig:lux_ana_lamp}. When changing the angle of the analyzer, we defined a positive and negative direction, which was kept for all subsequent measurements using polarization filters. The angle of the polarization filter was changed in $10^\circ$ increments in the range $-90^\circ$ to $90^\circ$ and the intensity of the light measured by the luxmeter was noted for each angle. 


  \begin{figure}[H]
    \center
    \includegraphics[width=8cm]{scripts/figs/diagram_1.png}
    \caption{Aparature to test the polarization of light emitted from a spectral lamp using a polarization filter with variable angle $\theta$ as an analyzer and measuring the intensity of the filtered light using the luxmeter.}
    \label{fig:lux_ana_lamp}
  \end{figure}

  \subsection{Testing Malus' law}

    In order to test Malus' law experimentally, we added a second polarization filter to our existing aperature (see sect. \ref{subsect:polar_lamp}), setting the first polarization filter to a fixed angle at $0^\circ$ whilst the second one is used as the analyzer, depicted in Fig. \ref{fig:lux_ana_pola_lamp}. In order to minimize the effects that ambient light and other potential contaminants, the apparatus was kept as tightly stacked as possible. The intensity registered by the luxmeter was again measured in $10^\circ$ increments in the range $-90^\circ$ to $90^\circ$ which was noted in the lab journal. 

    \begin{figure}[H]
      \center
      \includegraphics[width=8cm]{scripts/figs/diagram_2.png}
      \caption{Aparature to test Malus' law, where the polarizator is kept at a constant angle while the analyzer is varied.}
      \label{fig:lux_ana_pola_lamp}
    \end{figure}

    Afterwards, a third polarization filter was placed after the analyzer and set to a fixed angle at $90^\circ$ (see Fig. \ref{fig:lux_pola_ana_pola_lamp}). The angle of the analyzer was once again varied as before, and the intensity recorded by the luxmeter was noted in the journal.

    \begin{figure}[H]
      \center
      \includegraphics[width=8cm]{scripts/figs/diagram_3.png}
      \caption{Alternative aparature to test Malus' law, where the two polarizators are kept at a constant angle while the analyzer is varied.}
      \label{fig:lux_pola_ana_pola_lamp}
    \end{figure}

    \subsection{Reflection of polarized light}
      In order to test how the intensity of S and P-polarized light, reflected from a prism with a varying angle of incidence, a modified spectroscope was used as shown in Fig. \ref{fig:mod_spectro}. the angle, and intensity was recorded digitally on a computer using capstone. The angle recorder only measured the change in angle from starting the recording, so there was a notable uncertainty in zeroing the apparatus before starting. The sights, used to read the angle manually were blocked by a cable, such that it could not be used properly, and the geometry of the spectrometer itself was not perfect. For large angles, i.e when the incident line is nearly parallel with the plane of reflection, it was observed that some of the light was transmitted through the prism rather than being reflected. This is likely due to the prism not being properly aligned with the laser, which was reflected in the results. As such, the data for large angles $\phi$ do not follow what is expected, which must be taken into account when reading the results.

      As for making the measurements, the polarisator was set either parallel or tangential to the reflective plane of the prism and the prism was adjusted such that the reflective plane was paralell to the laser beam. Captstone was then started and the luxmeter and prism were slowly rotated untill the angle between the laser and the luxmeter was roughly $65^\circ$. This was repeated for both S and P-polarized light.

    \begin{figure}[H]
      \center
      \includegraphics[width=8cm]{scripts/figs/modified_spectrometer.png}
      \caption{Modified spectrometer used to test how the intensity of reflected, polarized light changes with the angle of incidence.}
      \label{fig:mod_spectro}
    \end{figure}

  \subsection{Observing birefringence in crystalline Calcite}
    We had two samples of crystalline calcite. One of which was naturally occurring, and another which was cut tangentially to its optical axis on two opposing sides. Both of the crystals were transparent\footnote{I believe the particular type of calcite used is named Iceland spar, or at the very least, it looks quite similar to it. See \url{https://en.wikipedia.org/wiki/Iceland_spar}}, and text was clearly readable when looking through them. The crystals were placed on an illuminated panel, on top of which was some text printed on transparent plastic. We then observed the text by looking through the two calcite crystals, both with and without a polarization filter acting as an Analyzer. 


\section{\label{sect:results}Results}
  
  The intensity measurements presented in table \ref{tab:ana}, where light from a spectral lamp is passed through a single polarization filter has a standard deviation of 19, which is used as the estimated uncertainty for further measurements made with the luxmeter.


  \begin{table}[H]
      \center
      \caption{Measured intensity when passing unpolarized light through a single polarization filter, $\theta$ denoting the angle of the filter. Aparature depicted in Fig. \ref{fig:lux_ana_lamp}}
       \begin{tabular}{r | l}
        $\theta$ [deg] & Intensity [Lux] \\ \hline
         \input{scripts/data/ana.dat}
       \end{tabular}
       \label{tab:ana}
  \end{table}



  \begin{figure}[H]
    \center
    \includegraphics[width=8cm]{scripts/ppolar.png}
    \caption{Intensity profile due to p-polarized light, where $\phi_P$ denotes the Brewster angle. Note that the measurements for angles prior to the peak are to be negated as discussed in the experimental section.}
  \end{figure}

  \begin{figure}[H]
    \center
    \includegraphics[width=8cm]{scripts/spolar.png}
    \caption{Intensity profile due to p-polarized light from two separate attempts of the experiment. Note that the measurements for angles prior to the peak are to be negated as discussed in the experimental section.}
  \end{figure}

\section{\label{sect:discuss}Discussion}

\section{\label{sect:conclusion}Conclusion}

\onecolumn

\bibliographystyle{plain}
\bibliography{references_rep6.bib}

%%%%%%%%%%%%%%%%%%%%%%%%
%%% END OF MAIN BODY %%%
%%%%%%%%%%%%%%%%%%%%%%%%
\end{document}