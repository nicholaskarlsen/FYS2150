              
% ****** Start of file apssamp.tex ******
%
%   This file is part of the APS files in the REVTeX 4.1 distribution.
%   Version 4.1r of REVTeX, August 2010
%
%   Copyright (c) 2009, 2010 The American Physical Society.
%
%   See the REVTeX 4 README file for restrictions and more information.
%
% TeX'ing this file requires that you have AMS-LaTeX 2.0 installed
% as well as the rest of the prerequisites for REVTeX 4.1
%
% See the REVTeX 4 README file
% It also requires running BibTeX. The commands are as follows:
%
%  1)  latex apssamp.tex
%  2)  bibtex apssamp
%  3)  latex apssamp.tex
%  4)  latex apssamp.tex
%
\documentclass[%
 reprint,
%superscriptaddress,
%groupedaddress,
%unsortedaddress,
%runinaddress,
%frontmatterverbose, 
%preprint,
%showpacs,preprintnumbers,
%nofootinbib,
%nobibnotes,
%bibnotes,
 amsmath,amssymb,
 aps,
%pra,
%prb,
%rmp,
%prstab,
%prstper,
%floatfix,
]{revtex4-1}

\usepackage{pgfplotstable}
\usepackage{array}
\usepackage{graphicx}% Include figure files
\usepackage{dcolumn}% Align table columns on decimal point
\usepackage{bm}% bold math
\usepackage{hyperref}% add hypertext capabilities
%\usepackage[mathlines]{lineno}% Enable numbering of text and display math
%\linenumbers\relax % Commence numbering lines

%\usepackage[showframe,%Uncomment any one of the following lines to test 
%%scale=0.7, marginratio={1:1, 2:3}, ignoreall,% default settings
%%text={7in,10in},centering,
%%margin=1.5in,
%%total={6.5in,8.75in}, top=1.2in, left=0.9in, includefoot,
%%height=10in,a5paper,hmargin={3cm,0.8in},
%]{geometry}
\bibliographystyle{plain}

%\graphicspath{{C:\Users\Nick\Documents\GitHub\FYS2150\lab1}}

\begin{document}

%\preprint{APS/123-QED}

\title{FYS2150 \\
Lab Report: Time and Frequency}% Force line breaks with \\

\author{Nicholas Karlsen}
% \email{nichoka@student.matnat.uio.no}

\date{\today}% It is always \today, today,
             %  but any date may be explicitly specified

\begin{abstract}
...
\end{abstract}

\maketitle

%\tableofcontents

\section{\label{sec:intro}Introduction}

\section{\label{sec:theory}Theory}

\begin{equation}
    T \approx 2\pi \sqrt{\frac{L}{g}}    
\end{equation}
$\pm$ 15s per day accuracy for $\theta < 1 rad$ according to wikipedia ref to wiki

\section{\label{sec:data} Results}

\begin{table}
\centering
\pgfplotstabletypeset[%
   fixed zerofill,
   precision=4,
   col sep=space,
   dec sep align,
   columns/0/.style ={column name=1},
   columns/1/.style ={column name=2},
   columns/2/.style ={column name=3},
   columns/3/.style ={column name=4},
]{scripts/lengdedata.csv}
\end{table}
    
\end{document}
%
% ****** End of file apssamp.tex ******
              