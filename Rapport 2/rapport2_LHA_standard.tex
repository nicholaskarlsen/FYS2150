\documentclass[11pt,a4paper]{article}
\usepackage{float}
\usepackage{verbatim}
\usepackage{subfig}
\usepackage[T1]{fontenc}
\usepackage[utf8]{inputenc}
\usepackage{geometry}
\usepackage{enumitem}
\geometry{verbose,lmargin=2cm,rmargin=2cm, bmargin=2cm, tmargin=2cm}
\usepackage{wrapfig}
\usepackage{tikz}
\usetikzlibrary{decorations.markings}
\usepackage{calc}
\usepackage{wrapfig}
\usepackage{graphicx}
\usepackage{amssymb}
\usepackage{amsmath}
\usepackage{esint}
\usepackage{hyperref}
\begin{document}

%\preprint{APS/123-QED}

\title{Lab Report: Length, Velocity and Acceleration}% Force line breaks with \\

\author{Nicholas Karlsen}
% \email{nichoka@student.matnat.uio.no}

\date{\today}% It is always \today, today,
             %  but any date may be explicitly specified

\maketitle

%\tableofcontents
\begin{abstract}
  A study on different methods for determining the length, velocity and acceleration of different objects, and the errors involved in these methods.
\end{abstract}
\section{\label{sec:intro}Introduction}

\section{\label{sec:theory}Theory}
  \subsection{Pendulum}
    \begin{equation}
      \label{eqn:period}
        T \approx 2\pi \sqrt{\frac{L}{g}}\enspace
    \end{equation}
    Where $T$ denotes the period of a pendulum, $L$ its length and $g$ the gravitational acceleration. The small angle approximation (Eqn. \ref{eqn:period})  is valid for angles $\theta \ll 1\, \textup{rad}$ with an error $\approx \pm 15 \, \textup{s}$ per day \cite{pend_wik}.
  \subsection{Errors}
    \begin{equation}
    \label{eqn:sigma}
      \sigma \approx \left(
      \frac{\sum x_i^2 - \frac{1}{n}(\sum x_i)^2}
      {n - 1}
      \right)^\frac{1}{2}
    \end{equation}
    
    \begin{equation}
      \label{eqn:sigma_m}
      \sigma_m \approx \left(
      \frac{\sum x_i^2 - \frac{1}{n}(\sum x_i)^2}
      {n(n - 1)}
      \right)^\frac{1}{2}
    \end{equation}

    Where $\sigma, \sigma_m$ denotes the standard deviation, and the standard deviation of the mean respectively of a set of $n$ values $x_i$. \cite{squires}.

    Any errors stated in a derived number will be calculated using the equations for combinations of errors found on page 29 in Squires \cite{squires}. Lastly, when using a linear fit on a set of linearly correlated data i used the expressions found on page 39 in Squires \cite{squires} to calculate the regression line, as well as its error.

\section{\label{sec:exp_proced}Experimental Procedure}
 
  \subsection{Measuring the difference in lengths between two rods}
    \subsubsection{Measurements using the Hultafors meter ruler}
      The rod was placed on a flat surface, and the end of the rod was lined up with the $1cm$ marker on the Hultafors Meter ruler \cite{hultafors} in order to negate any effect the "wear and tear" of the ruler might have on the results. This $1cm$ difference was accounted for in our reading of the data. The ruler was laid down on the table along with the rod, and did flex slightly because of this. The error due to flex is accounted for in the error section of the data sheet provided by the manufacturer. This procedure was repeated for both rods a and b.

    \subsubsection{Measurements using the Bosch PLR30}
      The rods were placed and secured to the table using adhesive tape on a table whilst being in direct contact with a wall. The Bosch PLR30 Distance measurer \cite{PLR}, henceforth referred to as laser, was then placed at the opposing end of the rod in order to measure the length from there, to the wall. Since the rod was touching the wall, this effectively means that we measured the length of the rod. There was a slight degree of systematic error in our procedure, as we could not ensure that the laser was pointing with exact parallel to the rod, nor did we have an exact way of placing the laser such that its origin would be at the exact end-point of the rod. 

    \subsection{Measurement using a digital vernier caliper}
      In order to determine the difference in length between the two rods directly we used a digital vernier caliper \cite{cocraft}. The rods were secured to a table in parallel, right next to each other with the ends on one side lined up with each other. The measurement of the difference in their lengths was then made on the other side using the vernier caliper. The vernier caliper was held above the two rods, resting on them in order to minimize any systematic error.

  \subsection{Measuring the period and height of the Foucault's pendulum}
    \subsubsection{Measuring the period of the pendulum}
    The measurements were taken in sequence using the lap function of the Cielo 100MT \cite{cielo} stopwatch. The time was recorded every other apex of the swing, which amounts to one period. All of the measurements were taken by the same person in order to ensure that the error in judgment and reaction time would remain the same throughout all of the measurements.

  \subsection{Measuring the velocity of the lego-car}
  
  
  \subsection{Measuring the velocity of the RC-car}

\section{\label{sec:data} Results}
  \begin{table}[H]
    \center
    \caption{Lenght of rods}
    \label{tab:lenrods}
    \begin{tabular}{rrrrrrrrr}
\hline
Ruler & Ruler & Ruler & Ruler & Laser & Laser & Laser & Laser & Vernier Calliper \\
   $l_a [cm]$ &  $\delta l_a$ [cm] &   $l_b$ [cm] &   $\delta l_b$ [cm] &   $l_a$ [cm] &   $\delta l_a$ [cm] & $l_b$ [cm] &   $\delta l_b$ [cm] & $l_{a, b}$ [mm] \\
\hline
           119.50 &              0.23 &           119.60 &              0.23 &           120.50 &              0.20 &           120.60 &              0.20 &                         1.25 $\pm 0.05$ \\
           119.50 &              - &           119.70 &              - &           119.60 &              0.20 &           119.80 &              0.20 &                         - \\
           119.45 &              0.37 &           119.60 &              0.37 &           119.50 &              0.20 &           119.70 &              0.20 &                         1.40 $\pm 0.05$\\
           119.40 &              - &           119.50 &              - &           119.40 &              0.20 &           119.60 &              0.20 &                         - \\
           119.43 &              0.40 &           119.55 &              0.40 &           119.40 &              0.20 &           119.60 &              0.20 &                         1.20 $\pm $0.6\\
           119.40 &              0.20 &           119.60 &              0.20 &           119.68 &              0.20 &           119.72 &              0.20 &                         1.80 $\pm $0.05\\
           119.40 &              0.27 &           119.50 &              0.27 &           119.90 &              0.20 &           119.70 &              0.20 &                         0.00 $\pm 0.05$\\
           119.45 &              0.35 &           119.65 &              0.35 &           130.60 &              0.20 &           130.20 &              0.20 &                         1.80 $\pm $0.05\\
           119.40 &              - &           119.60 &              - &           119.40 &              0.22 &           119.50 &              0.22 &                         - \\
           119.43 &              0.31 &           119.55 &              0.31 &             - &              - &             - &              - &                         1.50 $\pm 0.05$\\
\hline
\end{tabular}

  \end{table}

  Table. \ref{tab:lenrods} contains all the measurements made of the rods by the tuesday group, copied from the \href{https://uio.instructure.com/courses/910/modules/items/13592}{image} posted on canvas. Some of the data was not clearly readable, and has therefore been omitted from this table.

  \begin{table}[H]
    \center
    \caption{Uncertainty in Length measurement using the meter ruler}
    \label{tab:uncert}
    \begin{tabular}{| l | l | l |}
\hline
 & $x$ & $\delta x$ \\
\hline
$l_a$                 & $119.5$cm &    \\             
$l_b$                 & $119.6$cm &    \\        
$dl_s$                &     & $1.4$mm   \\             
$\sqrt{n}\cdot dl_l$  &     & $0.5\sqrt{5}$mm   \\                  
$dl_m$                &     & $1.4$mm   \\                  
$\alpha l_a (T-25C)$  & $-0.156$cm & $\sim 10^{-6}$mm   \\ 
\hline                

\end{tabular}

\begin{tabular}{| l | l | l |}
\hline
& $\sum x_i$ & $\sqrt {\sum \sigma x_i^2 }$ \\
\hline
$\sum l_a$ & 119.48cm & 2.27mm \\
$\sum l_b$ & 119.58cm & 2.27mm \\
\hline   
\end{tabular}
  \end{table}
  
  \begin{itemize}
    \item $l_a, \enspace l_b:$ Recorded length of rod $a$ and $b$ respectively
    \item $dl_s:$ Error due to aiming of the ruler
    \item $\sqrt n \cdot dl_l:$ Error due to curvature of joints
    \item $dl_m:$ Error due to precision of measuring lines
    \item $\alpha :\enspace 4\cdot10^{-5\circ} C^{-1}$, Coefficient of linear thermal expansion for glass fiber
  \end{itemize}
  
  \begin{table}[H]
    \center
    \caption{Period of pendulum}
    \label{tab:pendel}
    \begin{tabular}{rrrr}
\hline
   T [s]\\
\hline
    7.30\\ 7.72\\ 7.57\\ 7.43\\ 7.73\\ 7.27\\ 7.68\\ 7.60\\ 7.34\\ 7.75\\
    7.06\\ 7.32\\ 7.55\\ 7.29\\ 7.08\\ 7.82\\ 7.78\\ 7.44\\ 7.68\\ 7.46\\
\hline
\end{tabular}

  \end{table}
  \begin{figure}[H]
    \center
    \includegraphics[scale=0.7]{scripts/figs/period.png}
    \caption{Measurements of the Period of the Focault's Pendulum in the entrance hall at the Institute of Physics, UiO.}
    \label{fig:pendel}
  \end{figure}


\section{\label{sec:disc}Discussion}
\section{\label{sec:conc}Conclusion}


\begin{thebibliography}{1}

\bibitem{pend_wik}
\url{https://en.wikipedia.org/wiki/Pendulum}.

\bibitem{squires}
G.~L. Squires.
\newblock {\em Practical Physics 4th Edition}.
\newblock Cambridge University Press, 2001.

\bibitem{cocraft}
\url{http://www.uio.no/studier/emner/matnat/fys/FYS2150/v18/kursmateriell/datablader-og-brukermanualer/cocraft-digitalskyvel%C3%A6r.pdf}

\bibitem{hultafors}
\url{http://www.uio.no/studier/emner/matnat/fys/FYS2150/v18/kursmateriell/datablader-og-brukermanualer/hultafors_meterstokk.pdf}

\bibitem{PLR}
\url{http://www.uio.no/studier/emner/matnat/fys/FYS2150/v18/kursmateriell/datablader-og-brukermanualer/bosch_plr30.pdf}

\bibitem{cielo}
\url{http://www.uio.no/studier/emner/matnat/fys/FYS2150/v18/kursmateriell/datablader-og-brukermanualer/stoppeklokke.pdf}

\end{thebibliography}

\end{document}