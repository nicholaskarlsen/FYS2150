              
% ****** Start of file apssamp.tex ******
%
%   This file is part of the APS files in the REVTeX 4.1 distribution.
%   Version 4.1r of REVTeX, August 2010
%
%   Copyright (c) 2009, 2010 The American Physical Society.
%
%   See the REVTeX 4 README file for restrictions and more information.
%
% TeX'ing this file requires that you have AMS-LaTeX 2.0 installed
% as well as the rest of the prerequisites for REVTeX 4.1
%
% See the REVTeX 4 README file
% It also requires running BibTeX. The commands are as follows:
%
%  1)  latex apssamp.tex
%  2)  bibtex apssamp
%  3)  latex apssamp.tex
%  4)  latex apssamp.tex
%
\documentclass[%
 reprint,
%superscriptaddress,
%groupedaddress,
%unsortedaddress,
%runinaddress,
%frontmatterverbose, 
%preprint,
%showpacs,preprintnumbers,
%nofootinbib,
%nobibnotes,
%bibnotes,
 amsmath,amssymb,
 aps,
%pra,
%prb,
%rmp,
%prstab,
%prstper,
%floatfix,
]{revtex4-1}

\usepackage{graphicx}% Include figure files
\usepackage{dcolumn}% Align table columns on decimal point
\usepackage{bm}% bold math
%\usepackage{hyperref}% add hypertext capabilities
\usepackage{url}
%\usepackage[mathlines]{lineno}% Enable numbering of text and display math
%\linenumbers\relax % Commence numbering lines
\usepackage{listings}
\lstset{ %
  basicstyle=\footnotesize,        % the size of the fonts that are used for the code
  breakatwhitespace=false,         % sets if automatic breaks should only happen at whitespace
  breaklines=true,                 % sets automatic line breaking
  captionpos=t,                    % sets the caption-position to bottom
  deletekeywords={...},            % if you want to delete keywords from the given language
  escapeinside={\%*}{*)},          % if you want to add LaTeX within your code
  extendedchars=true,              % lets you use non-ASCII characters; for 8-bits encodings only, does not work with UTF-8
  frame=single,                    % adds a frame around the code
  keepspaces=true,                 % keeps spaces in text, useful for keeping indentation of code (possibly needs columns=flexible)
 % language=Python,                 % the language of the code
  morekeywords={*,...},           % if you want to add more keywords to the set
  numbers=left,                    % where to put the line-numbers; possible values are (none, left, right)
  numbersep=5pt,                   % how far the line-numbers are from the code
  showspaces=false,                % show spaces everywhere adding particular underscores; it overrides 'showstringspaces'
  showstringspaces=false,          % underline spaces within strings only
  showtabs=false,                  % show tabs within strings adding particular underscores
  stepnumber=1,                    % the step between two line-numbers. If it's 1, each line will be numbered
  tabsize=2,                       % sets default tabsize to 2 spaces
  title=\lstname                   % show the filename of files included with \lstinputlisting; also try caption instead of title
}


%\usepackage[showframe,%Uncomment any one of the following lines to test 
%%scale=0.7, marginratio={1:1, 2:3}, ignoreall,% default settings
%%text={7in,10in},centering,
%%margin=1.5in,
%%total={6.5in,8.75in}, top=1.2in, left=0.9in, includefoot,
%%height=10in,a5paper,hmargin={3cm,0.8in},
%]{geometry}
\bibliographystyle{plain}

%\graphicspath{{C:\Users\Nick\Documents\GitHub\FYS2150\lab1}}

\begin{document}

%\preprint{APS/123-QED}

\title{FYS2150 \\ Lab Report: Elasticity}% Force line breaks with \\

\author{Nicholas Karlsen}
% \email{nichoka@student.matnat.uio.no}

\date{\today}% It is always \today, today,
             %  but any date may be explicitly specified

\begin{abstract}
    A study on two different methods to determine the Young's modulus of a brass rod.
\end{abstract}

\maketitle

%\tableofcontents

\section{\label{sect:intro}Introduction}
    
\section{\label{sect:theory}Theory}
  \subsection{Euler-Bernoulli beam theory}
  \begin{equation}
    h(m) = \frac{mgl^3}{48EI}
  \end{equation}
  \begin{equation}
    E = \frac{4l^3g}{3\pi |A|d^4}
  \end{equation}
  \cite{wiki:euler_bernouli}
  \subsection{Errors}
  \cite{squires}

\section{\label{section:experimental}Experimental Procedure}
  \subsection{Three-point flexural test}
  \subsection{Measuring the speed of sound in the rod}

\section{\label{sect:results}Results}

\section{\label{sect:discussion}Discussion}
 
\section{\label{sect:conclusion}Conclusion}

%%%%%%%%%%%%%%%%%%%%%%%%
%%% END OF MAIN BODY %%%
%%%%%%%%%%%%%%%%%%%%%%%%

\bibliography{rapport3_ref}


\onecolumngrid % Start single collumn
\newpage % Ensure Code starts on new page
\section*{Code}
All of the code used to produce this report. Anything noteworthy should already be mentioned in the main body of the report.
\lstinputlisting[language=python]{scripts/lab_data.py}
\lstinputlisting[language=python]{scripts/FYS2150lib.py}

\twocolumngrid % Stop single collumn

\end{document}
